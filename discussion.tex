\chapter{Discussion} \label{discussion}


\section{Project Limitations}

Audio feedback is the major factor that limits the usability of our device. Without any notch filter in place the unstable feedback makes the device unusable. With a passive twin-T notch filter we were able to minimise the effect of this unstable feedback so that it only occurred when the stethoscope head was placed within 10cm of the speaker. This could possibly be enough to eliminate the feedback as the speaker could be placed outside of a 10cm radius from the patient while a doctor performs auscultations. The notch filter itself is a crude first attempt to address the instability, and only showed that a notch filter of some sort can potentially make the stethoscope useful. The design isn't feasible to use in a final product because of its large bandwidth of 800Hz (Fig.~\ref{fig:notch_sim}), which attenuates a large portion of frequencies within the range of auscultation signals. Additionally it is not tunable, once capacitor and resistor values are chosen for the notch filter the hardware determines the notched frequency. Although it's possible to introduce a variable resistor or capacitor to tune this cut out frequency this is not a desired solution as it would require a practitioner to endure an unpleasant feedback screech while trying to tune the device. Furthermore the unstable feedback frequency may change as a result of the environment or slight variations in manufacturing the device. A better solution may be to use the microprocessor to digitally filter the unstable frequencies, which would allow for tunability and the possibility of dynamically retuning the notch filter.

It appears that the unstable feedback is not a result from the electrical circuit, as no distinct resonances were found in the sensor amplification or circuit input stages (Fig.~\ref{fig:result_circuit_freq_response}). This and the fact that the feedback frequencies changed when we used a different coupling device to connect the stethoscope to the microphone suggest that an acoustic resonance is the root of the instability. Thus we need to test our  device in the working environment within the Royal Melbourne Hospital where it will be used, to account for the possibility that this room may have its own resonances associated with it. Furthermore it may be subject to different sources of noise not encountered in the lab we developed our device, such as medical equipment or room ventilation fans, that could cause other instabilities.

Currently we can hear our own heart beat with the stethoscope placed on our chest, however we have no method for determining the quality of sound the stethoscope produces. We don't know if the sound quality is good enough and so need to devise an experimental procedure that would give a measure of how acceptable the sound quality is. This would need to be done in collaboration with doctors to get feedback on what sound qualities they are listening for and whether the device meets them.
	

\section{Future Development}

To address the issue of feed back instability further work could be undertaken to place sound absorbing materials around the microphone, such as silicon or a porous fibre, in order to minimise the amount of feedback sound entering the microphone from directions other than the stethoscope diaphragm. The issue with using a microphone sensor is the fact that it picks up all sources of sound nearby, not just the auscultation sounds coming from the stethoscope diaphragm. It is possible that this could be entirely eliminated by using one of the other sensors proposed, such as a capacitive or laser sensor. By using either of these the measured signal comes directly from the stethoscope diaphragm. If the system speaker does not appreciably affect this diaphragm then it is possible that no feedback treatment will be needed at all. It's reasonable to think that a nearby speaker wouldn't perturb the diaphragm significantly because of it's relatively large surface area and mass, however only further experimentation can determine if this is the case. To continue investigation with one of these sensors may also require either building very small circuitry to fit inside the stethoscope head, finding commercially available stethoscopes with relatively larger heads, or fabricating an entirely new diaphragm, essentially building the stethoscope from scratch.

Alternatively if a microphone sensor is to be used then the issue with feedback could be addressed in several ways. Firstly the micro-controller could be used to implement a digital notch filter. This notch filter could be made dynamic by frequently measuring the FFT of the input signal and adjusting filter coefficients as necessary to filter out any anomalously large frequency components, which would be assumed to have arisen from unstable feedback. This approach requires the DSP to calculate the coefficients itself, using an appropriate algorithm. Currently to implement a digital filter we would use software package such as MATLAB to calculate the required coefficients needed for a particular notch frequency. If this computation was to be moved to the micro-controller it may prove too computationally intensive for the dsPIC33 to handle in real time. 

An analog solution to the feed back problem would be to treat the signal at the input. An additional identical microphone could be placed outside the stethoscope, with both microphones being fed into a difference amplifier. This difference amplifier would then ideally subtract the unstable feedback frequencies from the input signal. This is because the microphone outside the stethoscope can only pick up signals from the speaker, while the microphone inside the stethoscope picks up both the speaker and stethoscope diaphragm signals, when both microphone signals are subtracted then only the desired stethoscope signal remains. There are potential issues with scaling the input from both microphones, as the feedback signal picked up by both needs to be of approximately equal intensity in order to cancel out. Furthermore the position of the microphones relative to each other may introduce a phase shift in the unstable feedback signal received from the speaker, leading to minimal cancellation of the undesired feedback. 


Currently our system transmits data from the micro-controller to the Blue tooth module via an analog output, as the module is designed to be used with an analog signal source such as a portable music playing device. This introduces another level of quantisation noise that could be avoided if this signal was sent digitally. Further work could be undertaken to reprogram the Blue tooth module to interface digitally. This would also require additional hardware as the header pins of the Blue tooth module only connect to power, ground, and analog input/output channels. 


Finally it was our intention to apply a digital filter that would mimic the sound characteristics of a traditional stethoscope, however due to time constraints and the more pressing issue of unstable feedback we did not implement this functionality. Further work needs to be done to identify the characteristic equation of the transfer function observed for the stethoscope tubing (Fig.~\ref{fig:result_tubing_freq_response}), so that its inverse can be identified and applied to an input signal. Furthermore it's desirable to measure the sound transmission frequency response of the body suits doctors will wear while inside the isolation rooms, so that their effect on sound can be characterised and treated for in a similar manner. Such filters may require more computation then currently feasible on the dsPIC, however to increase the computational power per sample the sampling frequency of the system can be reduced down to 8kHz, in order to only pick up sound frequencies below 4kHz that compose most auscultation sounds (Section~\ref{max-sound-freq}). This would allow for more computations to be carried out between each sample and thus higher order filters to be implemented. With higher order filters we would be given more flexibility in treating for the acoustic effects of elements such as the tube length and isolation suits. 
